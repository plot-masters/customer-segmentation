% Options for packages loaded elsewhere
% Options for packages loaded elsewhere
\PassOptionsToPackage{unicode}{hyperref}
\PassOptionsToPackage{hyphens}{url}
\PassOptionsToPackage{dvipsnames,svgnames,x11names}{xcolor}
%
\documentclass[
  letterpaper,
  DIV=11,
  numbers=noendperiod]{scrartcl}
\usepackage{xcolor}
\usepackage{amsmath,amssymb}
\setcounter{secnumdepth}{-\maxdimen} % remove section numbering
\usepackage{iftex}
\ifPDFTeX
  \usepackage[T1]{fontenc}
  \usepackage[utf8]{inputenc}
  \usepackage{textcomp} % provide euro and other symbols
\else % if luatex or xetex
  \usepackage{unicode-math} % this also loads fontspec
  \defaultfontfeatures{Scale=MatchLowercase}
  \defaultfontfeatures[\rmfamily]{Ligatures=TeX,Scale=1}
\fi
\usepackage{lmodern}
\ifPDFTeX\else
  % xetex/luatex font selection
\fi
% Use upquote if available, for straight quotes in verbatim environments
\IfFileExists{upquote.sty}{\usepackage{upquote}}{}
\IfFileExists{microtype.sty}{% use microtype if available
  \usepackage[]{microtype}
  \UseMicrotypeSet[protrusion]{basicmath} % disable protrusion for tt fonts
}{}
\makeatletter
\@ifundefined{KOMAClassName}{% if non-KOMA class
  \IfFileExists{parskip.sty}{%
    \usepackage{parskip}
  }{% else
    \setlength{\parindent}{0pt}
    \setlength{\parskip}{6pt plus 2pt minus 1pt}}
}{% if KOMA class
  \KOMAoptions{parskip=half}}
\makeatother
% Make \paragraph and \subparagraph free-standing
\makeatletter
\ifx\paragraph\undefined\else
  \let\oldparagraph\paragraph
  \renewcommand{\paragraph}{
    \@ifstar
      \xxxParagraphStar
      \xxxParagraphNoStar
  }
  \newcommand{\xxxParagraphStar}[1]{\oldparagraph*{#1}\mbox{}}
  \newcommand{\xxxParagraphNoStar}[1]{\oldparagraph{#1}\mbox{}}
\fi
\ifx\subparagraph\undefined\else
  \let\oldsubparagraph\subparagraph
  \renewcommand{\subparagraph}{
    \@ifstar
      \xxxSubParagraphStar
      \xxxSubParagraphNoStar
  }
  \newcommand{\xxxSubParagraphStar}[1]{\oldsubparagraph*{#1}\mbox{}}
  \newcommand{\xxxSubParagraphNoStar}[1]{\oldsubparagraph{#1}\mbox{}}
\fi
\makeatother


\usepackage{longtable,booktabs,array}
\usepackage{calc} % for calculating minipage widths
% Correct order of tables after \paragraph or \subparagraph
\usepackage{etoolbox}
\makeatletter
\patchcmd\longtable{\par}{\if@noskipsec\mbox{}\fi\par}{}{}
\makeatother
% Allow footnotes in longtable head/foot
\IfFileExists{footnotehyper.sty}{\usepackage{footnotehyper}}{\usepackage{footnote}}
\makesavenoteenv{longtable}
\usepackage{graphicx}
\makeatletter
\newsavebox\pandoc@box
\newcommand*\pandocbounded[1]{% scales image to fit in text height/width
  \sbox\pandoc@box{#1}%
  \Gscale@div\@tempa{\textheight}{\dimexpr\ht\pandoc@box+\dp\pandoc@box\relax}%
  \Gscale@div\@tempb{\linewidth}{\wd\pandoc@box}%
  \ifdim\@tempb\p@<\@tempa\p@\let\@tempa\@tempb\fi% select the smaller of both
  \ifdim\@tempa\p@<\p@\scalebox{\@tempa}{\usebox\pandoc@box}%
  \else\usebox{\pandoc@box}%
  \fi%
}
% Set default figure placement to htbp
\def\fps@figure{htbp}
\makeatother


% definitions for citeproc citations
\NewDocumentCommand\citeproctext{}{}
\NewDocumentCommand\citeproc{mm}{%
  \begingroup\def\citeproctext{#2}\cite{#1}\endgroup}
\makeatletter
 % allow citations to break across lines
 \let\@cite@ofmt\@firstofone
 % avoid brackets around text for \cite:
 \def\@biblabel#1{}
 \def\@cite#1#2{{#1\if@tempswa , #2\fi}}
\makeatother
\newlength{\cslhangindent}
\setlength{\cslhangindent}{1.5em}
\newlength{\csllabelwidth}
\setlength{\csllabelwidth}{3em}
\newenvironment{CSLReferences}[2] % #1 hanging-indent, #2 entry-spacing
 {\begin{list}{}{%
  \setlength{\itemindent}{0pt}
  \setlength{\leftmargin}{0pt}
  \setlength{\parsep}{0pt}
  % turn on hanging indent if param 1 is 1
  \ifodd #1
   \setlength{\leftmargin}{\cslhangindent}
   \setlength{\itemindent}{-1\cslhangindent}
  \fi
  % set entry spacing
  \setlength{\itemsep}{#2\baselineskip}}}
 {\end{list}}
\usepackage{calc}
\newcommand{\CSLBlock}[1]{\hfill\break\parbox[t]{\linewidth}{\strut\ignorespaces#1\strut}}
\newcommand{\CSLLeftMargin}[1]{\parbox[t]{\csllabelwidth}{\strut#1\strut}}
\newcommand{\CSLRightInline}[1]{\parbox[t]{\linewidth - \csllabelwidth}{\strut#1\strut}}
\newcommand{\CSLIndent}[1]{\hspace{\cslhangindent}#1}



\setlength{\emergencystretch}{3em} % prevent overfull lines

\providecommand{\tightlist}{%
  \setlength{\itemsep}{0pt}\setlength{\parskip}{0pt}}



 


\KOMAoption{captions}{tableheading}
\makeatletter
\@ifpackageloaded{caption}{}{\usepackage{caption}}
\AtBeginDocument{%
\ifdefined\contentsname
  \renewcommand*\contentsname{Table of contents}
\else
  \newcommand\contentsname{Table of contents}
\fi
\ifdefined\listfigurename
  \renewcommand*\listfigurename{List of Figures}
\else
  \newcommand\listfigurename{List of Figures}
\fi
\ifdefined\listtablename
  \renewcommand*\listtablename{List of Tables}
\else
  \newcommand\listtablename{List of Tables}
\fi
\ifdefined\figurename
  \renewcommand*\figurename{Figure}
\else
  \newcommand\figurename{Figure}
\fi
\ifdefined\tablename
  \renewcommand*\tablename{Table}
\else
  \newcommand\tablename{Table}
\fi
}
\@ifpackageloaded{float}{}{\usepackage{float}}
\floatstyle{ruled}
\@ifundefined{c@chapter}{\newfloat{codelisting}{h}{lop}}{\newfloat{codelisting}{h}{lop}[chapter]}
\floatname{codelisting}{Listing}
\newcommand*\listoflistings{\listof{codelisting}{List of Listings}}
\makeatother
\makeatletter
\makeatother
\makeatletter
\@ifpackageloaded{caption}{}{\usepackage{caption}}
\@ifpackageloaded{subcaption}{}{\usepackage{subcaption}}
\makeatother
\usepackage{bookmark}
\IfFileExists{xurl.sty}{\usepackage{xurl}}{} % add URL line breaks if available
\urlstyle{same}
\hypersetup{
  pdftitle={Customer Segmentation Model},
  colorlinks=true,
  linkcolor={blue},
  filecolor={Maroon},
  citecolor={Blue},
  urlcolor={Blue},
  pdfcreator={LaTeX via pandoc}}


\title{Customer Segmentation Model}
\author{Utkarsh Tripathi (2025EM1100146) \and Juwaria Qadri
(2025EM1100132) \and Mohammed Omar (2025EM1100235) \and Merin Ann
Cherian (2025EM1100211)}
\date{}
\begin{document}
\maketitle


\section{Phase 1}\label{phase-1}

Segmenting Blinkit customers based on their spending behaviour and
delivery experience to identify distinct customer clusters (high,
medium, and low spending) for targeted marketing strategies and
increasing sales

\subsection{Proposal}\label{proposal}

\begin{enumerate}
\def\labelenumi{\arabic{enumi}.}
\item
  \textbf{Project Statement}\\
  Across customer-centric businesses, one of the major goals for the
  organization is being able to predict the needs of its customers and
  how to best cater to their needs while maintaining high profitability.
  However, to make the solution feasible, the way forward would be to
  create groups of consumers that share similar spending patterns,
  allowing the businesses to create targeted marketing campaigns,
  improve customer satisfaction, demand-oriented product development,
  and therefore improve \emph{profitability}.
\item
  \textbf{Business Goal}\\
  The objective of this project is to develop a machine learning model
  that can create suitable and usable clusters/segments given the
  profile and purchases of the customer. The model should then be able
  to predict, within bounds of acceptable error, which segment a
  customer is likely to belong to. The model should allow stakeholders
  to make informed and data-backed decisions on product demand, campaign
  successes and customer retention.
\item
  \textbf{Data Source}\\
  We will use the ``Blinkit Sales Dataset'' for this project. This
  dataset provides detailed information on product sales, visibility,
  item types, and outlet performance, making it ideal for performing
  sales data analysis and gaining insights into business trends. The
  dataset is well-structured and suitable for data preprocessing,
  exploratory data analysis, and predictive modeling tasks.

  \begin{itemize}
  \tightlist
  \item
    \textbf{Source Platform:} \textbf{Kaggle}
  \item
    \textbf{Dataset}: Blinkit Sales Data (Vaghasiya 2025)
  \end{itemize}
\item
  \textbf{Tools and Technology}\\
  Following languages and libraries are planned to be used for this
  project. More libraries may be used and every non trivial library
  shall be updated.

  \begin{itemize}
  \tightlist
  \item
    Python

    \begin{itemize}
    \tightlist
    \item
      Core Libraries

      \begin{itemize}
      \tightlist
      \item
        Data Manipulation: Pandas, NumPy\\
      \item
        ML: scikit-learn\\
      \item
        Data Visualization \& Storytelling: Matplotlib, Seaborn\\
      \end{itemize}
    \item
      Development Environment: VS Code, Google Colab, GitHub\\
    \end{itemize}
  \item
    Dashboard: Power BI
  \end{itemize}
\item
  \textbf{Project Workflow}\\
  The project is to follow a standard data science development
  lifecycle,

  \begin{enumerate}
  \def\labelenumii{\arabic{enumii}.}
  \tightlist
  \item
    Data Acquisition: Fetch the dataset from \textbf{Kaggle} using its
    API.\\
  \item
    Preprocessing: Handle missing values (if any), encode categorical
    variables, and\\
    check for data inconsistencies.\\
  \item
    EDA: Analyze features to understand their relationship with
    attrition using statistical summaries and visualizations.\\
  \item
    Feature Engineering: Create new features from existing ones if
    necessary to improve model performance.\\
  \item
    Modeling: Train several clustering models (e.g., K Means Clustering,
    PCA, Decision Trees).\\
  \item
    Evaluation: Assess model performance using metrics like Accuracy,
    Precision, Recall, and F1-Score. Select the best-performing model.\\
  \item
    Visualization: Create an interactive dashboard in Power BI to
    present the key findings and predictions to stakeholders.
  \end{enumerate}
\item
  \textbf{Data Extraction}\\
  The ``\emph{Blinkit Sales Dataset}'' is acquired directly from the
  Kaggle repository. To ensure a professional and reproducible workflow,
  manually downloading the files is not done.\\
  Instead, we will perform the following steps:

  \begin{itemize}
  \tightlist
  \item
    Automate the Process: We will write a Python script that utilizes
    the official Kaggle API to connect to the source and download the
    dataset.\\
  \item
    Ensure Reproducibility: This scripted approach guarantees that the
    data extraction process is consistent and can be easily re-run by
    any team member or reviewer.\\
  \item
    Prepare for Analysis: The script will handle the unzipping of the
    downloaded files and load the data directly into a Pandas DataFrame,
    making it immediately available for the next phase of our project.\\
  \item
    Notebook: \href{notebook/customer_segmentation_model.ipynb}{Customer
    Segmentation Notebook}
  \end{itemize}
\item
  \textbf{Schema / Data Dictionary}\\
  This data dictionary was created after inspecting the dataset.
\end{enumerate}

\begin{longtable}[]{@{}lllll@{}}
\toprule\noalign{}
& Feature Name & Data Type & Description & PK (Yes/No) \\
\midrule\noalign{}
\endhead
\bottomrule\noalign{}
\endlastfoot
0 & order\_id & int64 & Unique identifier for each order & Yes \\
1 & customer\_id & int64 & Unique identifier for each customer & No \\
2 & order\_date & object & The timestamp when the order was placed &
No \\
3 & promised\_delivery\_time & object & The time informed to the
customer for completi... & No \\
4 & actual\_delivery\_time & object & The actual time when the order was
delivered & No \\
5 & delivery\_status & object & The delivery status of the order & No \\
6 & order\_total & float64 & The total price of the order (in Rupees) &
No \\
7 & payment\_method & object & The payment method used by the customer &
No \\
8 & delivery\_partner\_id & int64 & Unique identifier representing the
delivery pa... & No \\
9 & store\_id & int64 & Unique identifier for the store that
fulfilled... & No \\
\end{longtable}

\textsubscript{Source:
\href{https://plot-masters.github.io/customer-segmentation/notebook/data_dictionary.ipynb.html\#cff88bb5}{Data
Dictionary}}

\section{Phase 2}\label{phase-2}

\subsection{Data Extraction}\label{data-extraction}

\begin{verbatim}
<class 'pandas.core.frame.DataFrame'>
RangeIndex: 5000 entries, 0 to 4999
Data columns (total 10 columns):
 #   Column                  Non-Null Count  Dtype  
---  ------                  --------------  -----  
 0   order_id                5000 non-null   int64  
 1   customer_id             5000 non-null   int64  
 2   order_date              5000 non-null   object 
 3   promised_delivery_time  5000 non-null   object 
 4   actual_delivery_time    5000 non-null   object 
 5   delivery_status         5000 non-null   object 
 6   order_total             5000 non-null   float64
 7   payment_method          5000 non-null   object 
 8   delivery_partner_id     5000 non-null   int64  
 9   store_id                5000 non-null   int64  
dtypes: float64(1), int64(4), object(5)
memory usage: 390.8+ KB
\end{verbatim}

\begin{verbatim}
order_id                  0
customer_id               0
order_date                0
promised_delivery_time    0
actual_delivery_time      0
delivery_status           0
order_total               0
payment_method            0
delivery_partner_id       0
store_id                  0
dtype: int64
\end{verbatim}

\begin{longtable}[]{@{}ll@{}}
\toprule\noalign{}
& 0 \\
\midrule\noalign{}
\endhead
\bottomrule\noalign{}
\endlastfoot
order\_id & int64 \\
customer\_id & int64 \\
order\_date & object \\
promised\_delivery\_time & object \\
actual\_delivery\_time & object \\
delivery\_status & object \\
order\_total & float64 \\
payment\_method & object \\
delivery\_partner\_id & int64 \\
store\_id & int64 \\
\end{longtable}

\textsubscript{Source:
\href{https://plot-masters.github.io/customer-segmentation/notebook/customer_segmentation_model-preview.html\#cell-0}{customer\_segmentation\_model.ipynb}}

\phantomsection\label{refs}
\begin{CSLReferences}{1}{0}
\bibitem[\citeproctext]{ref-blinkit_2025}
Vaghasiya, Akshit. 2025. {``Blinkit {Sales} {Dataset}.''}
\url{https://www.kaggle.com/datasets/akxiit/blinkit-sales-dataset}.

\end{CSLReferences}




\end{document}
